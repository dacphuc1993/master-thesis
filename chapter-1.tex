%%
%% VERSION HISTORY
%%    22 May 2006 - John Papandriopoulos - Original version
%%    12 Jul 2007 - John Papandriopoulos - Converted into template
%%

\chapter{Introduction}

%=========================================================================

During the past decade, the number of Internet users has increased exponentially to nearly 3.2 billions users. A large portion of them use social media websites for a wide range of activities: from sharing personal posts, discussing a particular topic to advertizing businesses. These applications are enables by numerous advantages of social networking websites compared to traditional forms of communication namely convenient connectivity, easy to access and adaptability to its users. Among the leading social network sites, some can attract from hundred of million up to billions of users such as Facebook\footnote{https://www.facebook.com/} (2 billion monthly users \cite{Welch2017}), Twitter\footnote{https://twitter.com/} (328 million active users \cite{Aslam2017}) or Reddit\cite{https://www.reddit.com/} (234 million unique users \cite{Alexa2017}). Each of these websites has different main features and functions, but overall allows users to post their content publicly.\\
The rapid development of social media sites has changed the habits of perceiving information and express opinions of young generation. They are now more open to speak out their mind via computer-mediated communication channel. The shift to digital age has enabled researchers in mental health area to access data much more easily. It provides powerful tools to collect structured and unstructured data, opening directions in online public health such as mental health surveillance: number of suicide case predicting \cite{Won2013}, mental health-related posting behaviors \cite{Berry2017} or stress level of individuals monitoring \cite{Liu2017}. \\
There are a variety of approaches and scopes of studies since the emergence of both large scale data collection and data analysis and visualization method. At population level, observations using trend-based approach facilitated by big data platform or social network sites - for instance, Google Trend\footnote{https://trends.google.com/trends/ } allows user to visualize statistics of keyword(s) sorted by geographical area or time period - are proved be useful to develop prediction models for general population. \\
At individual level, types of study diversify even more. Many papers utilize Natural Language Processing (NLP) techniques to analyze the language pattern of a large number of user and identify people are vulnerable to mental health illness even before the onset of symptoms \cite{Almeida, DeChoudhury2013}. Another direction that aims to support individuals is intervention with involvement of technology. Numerous mobile and smartphone applications with features like personal reminder, providing relevant information for users have been developed to study the efficacy and feasibility of mobile intervention \cite{Rathbone2017}. \\
The increase in data storing capacity and data transfer speed also enhance the real time interaction between people. This leaf in technology encourages computer-mediated communication and make online therapies possible. Several organizations and technology start-ups companies build their service to connect people in need of mental health counseling to trained voluteers or mental health professionals (e.g. 7cups\footnote{https://www.7cups.com/}, CrisisTextLine\footnote{https://www.crisistextline.org/}, TalkLife\footnote{https://talklife.co/}). These sites ultilize the synchronous text-based online interventions - mainly online chat and text message - and generally obtain positive result in post-treatment reviews from users \cite{Hoermann2017}.\\
In the recent years, virtual reality technologies develop quickly and many potential applications like gaming,

however, the capability of VR and other intervention is limited, leaving text-based main method of treatment.......


The massive volume of content generated by users everyday facilitates 


