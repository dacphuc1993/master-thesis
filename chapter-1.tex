%%
%% VERSION HISTORY
%%    22 May 2006 - John Papandriopoulos - Original version
%%    12 Jul 2007 - John Papandriopoulos - Converted into template
%%

\chapter{Introduction}

%=========================================================================
\section{Problem background}
\subsection*{Internet and social networking sites}
During the past decade, the number of Internet users has increased exponentially to nearly 3.2 billions users. A large portion of them use social media websites for a wide range of activities: from sharing personal posts, discussing a particular topic to advertizing businesses. These applications are enables by numerous advantages of social networking websites compared to traditional forms of communication namely convenient connectivity, easy to access and adaptability to its users. Among the leading social network sites, some can attract from hundred of million up to billions of users such as Facebook\footnote{https://www.facebook.com/} (2 billion monthly users \cite{Welch2017}), Twitter\footnote{https://twitter.com/} (328 million active users \cite{Aslam2017}) or Reddit\cite{https://www.reddit.com/} (234 million unique users \cite{Alexa2017}). Each of these websites has different main features and functions, but overall allows users to post their content publicly.\\

The rapid development of social media sites has changed the habits of perceiving information and express opinions of young generation. They are now more open to speak out their mind via computer-mediated communication channel. The shift to digital age has enabled researchers in mental health area to access data much more easily. It provides powerful tools to collect structured and unstructured data, opening directions in online public health such as mental health surveillance: number of suicide case predicting \cite{Won2013}, mental health-related posting behaviors \cite{Berry2017} or stress level of individuals monitoring \cite{Liu2017}. \\

\subsection*{Youth mental health}
Mental health disorders and poor mental state have detrimental effects on one's life such as decreasing enjoyment and quality of life, lowering productivity, making one vulnerable to abuse and ultimately susceptible to suicide. The causes of mental illness come from psychological, biological, and environmental factors of each individual. Some factors cannot be controlled like genetic, individuals have higher risk of developing mental illness if their family history show symptoms of it, such factor is called risk factor. Heredity of mental illness has been observed in previous studies \cite{Hyman2000}. On the other hand, factors from environment - for example, education and affection in family are protective factors - can be modified by people. Risk factors are factors occur before symptoms of mental health problems and contribute to the risk of developing mental illness \cite{KAZDIN1997375}. In contrast, protective factors mitigate the effects of risk factor, preventing mental illness progressing \cite{shortt_spence_2006}. \\

For the past two decades, intervention programs that aim to promote youth mental health have been improved in many aspects. Before the Internet plays a vital role in information dissemination, these programs were implemented mostly in school settings considering that is the most effective ways to reach adolescents at school. However, the school-based approach requires a lot of human resources and intensive engagement with students. As a consequence, it is hard to sustain the program upon completion and there is high chance that funding withdrawal decision being issued when project was running halfway. Spence and Shortt \cite{JCPP:JCPP1738} have raised their concerns regarding effectiveness and efficacy of school-based programs. It might not be justified for the objective being preventing student from developing depression in long term while the programs were brief and did not incorporate environmental change to evaluate resilience of students. Other than school-based programs, face-to-face intervention also take place at medical centers and clinics with mental health professionals directly give diagnosis and instruction to service users. Naturally, those means of intervention have similar limitations in sustainability and implementation namely time-consuming, adequate training for instructors, cost for setting up and maintaining physical location.\\

The Internet evolution offers a new way of delivery for intervention programs. Now the programs can be delivered to more students without incurring fixed cost for each additional participants through communication channels provided by web servers. The number of Internet user under the age of 25 is increasing dramatically. The advantages of the Internet in self-learning, as a direct result of Information Retrieval, allow adolescents to access information more effectively without restrictions to space and time. Moreover, the cost for server maintenance is minimal when compared to costs of school-based setting programs, making it easier to sustain the program until its completion. Although the research approach using Internet-based therapies is pretty young, there is evidence indicating efficacy of Internet-based treatment programs is equal or better to programs with school setting. There are several web-based intervention programs that was deployed to enhance awareness of adolescents about mental health namely beyondblue\footnote{https://www.beyondblue.org.au/}, MyHealthMagazine\footnote{http://www.yoomagazine.net/}, WalkAlong\footnote{https://www.walkalong.ca/}, moodgym\footnote{https://moodgym.com.au/}. These websites' main feature is providing content and information related to mental health. They made the effort to present the information in an interactive and educational way because they expect visitors to be adolescents. Due to their short time of deployment, the effectiveness of most of those programs have not been assessed by research methodology, therefore we cannot draw conclusion about the exact weak points of Internet-based programs compared to school-based projects. However, if Internet intervention programs are proved to be as effective as one of traditional programs in term of preventing depression and psychological disorders, it is worth shifting programs from face-to-face delivery to Internet delivery because of mentioned advantages and characteristics of Internet-based programs \cite{Huen2016}.\\

\subsection*{Mental health on Internet}
There are a variety of approaches and scopes of studies since the emergence of both large scale data collection and data analysis and visualization method. At population level, observations using trend-based approach facilitated by big data platform or social network sites - for instance, Google Trend\footnote{https://trends.google.com/trends/ } allows user to visualize statistics of keyword(s) sorted by geographical area or time period - are proved be useful to develop prediction models for general population. \\

At individual level, types of study diversify even more. Many papers utilize Natural Language Processing (NLP) techniques to analyze the language pattern of a large number of user and identify people are vulnerable to mental health illness even before the onset of symptoms \cite{Almeida, DeChoudhury2013}. Another direction that aims to support individuals is intervention with involvement of technology. Numerous mobile and smartphone applications with features like personal reminder, providing relevant information for users have been developed to study the efficacy and feasibility of mobile intervention \cite{Rathbone2017}. \\

The increase in data storing capacity and data transfer speed also enhance the real time interaction between people. This leaf in technology encourages computer-mediated communication and make online therapies possible.Web-based apps are deployed and recognized for their cost-effectiveness in delivering therapies within reasonable time frame via computers. Web-based therapies have significant advantages over traditional face-to-face therapies like cognitive behavioral therapy (CBT). In the past, people must meet professionals in person and it poses barriers such as transportation, scheduling for appointments and long waiting time lead to high cost and inconvenience. Whereas web-based apps do not encounter any of mentioned issues, they offer services to much more larger percentage of population. Several organizations and technology start-ups companies build their service to connect people in need of mental health counseling to trained voluteers or mental health professionals (e.g. 7cups\footnote{https://www.7cups.com/}, CrisisTextLine\footnote{https://www.crisistextline.org/}, TalkLife\footnote{https://talklife.co/}). These sites ultilize the synchronous text-based online interventions - mainly online chat and text message - and generally obtain positive results in post-treatment reviews from users \cite{Hoermann2017}.\\

\section{Problem motivation}
\subsection*{Choice of data source}
The limitations of traditional clinical treatment have caused reluctance of young people to seek help from mental health professionals in face-to-face setting. The main concerns are financial cost, accessibility of services, lack of information and knowledge and social stigma associated with mental illness \cite{Rickwood2005}. Along with advancement of Internet, young people tend to adopt a habit of being online regularly. They find information about their mental health problems on the Internet and this has become a growing tendency among people experiencing depression and other psychological disorders. Furthermore, youngsters have particular interest for online help due to availability, high levels of anonymity, relatively low cost and less stigmatizing than conventional intervention. These advantages help young people to overcome barriers specified earlier when seeking suitable help for them.\\
We have reviewed some of Internet-based intervention programs, they fall into one of the three categories: \\
(1) Self-help: people seek knowledge regarding their symptoms. \\
(2) Professional treatment: patients connect to trained clinicians via websites designed for mental health topics. \\
(3) Peer-to-peer support: people connect to non-professionals for advices, personal stories and emotional support.\\
Among those three forms of intervention, peer-to-peer support is a prominent aspect both to Internet users and researchers. Previous study showed that millions of people visit support groups on daily basis. Peer-to-peer support gathers people with similar stressors or health problems with the aim of mutual support or assistance of experienced users to other users in need of help. Peer support takes place in pairs or groups via the Internet. There are many online peer support platforms, they are either asynchronous (e.g. discussion groups, forums) or synchronous (e.g. chat room).\\
Both paradigms are popular and widely used for intervention, however, they are different in nature. Chat-based intervention usually involves mental health client and a person or several people who act as counselors even though they are not necessarily mental health professionals. Conversations take place in real time via websites or mobile phone apps, helping listeners to engage with users and their stories in great details. Most of such services are very cautious about data of conversations due to confidentiality of users. During the chat, users often reveal personally identifiable information such as name, location or employment history. Each country has its own privacy law or act where leaking private information can lead to legal actions. Because of this sensitive matter, chat-based services always consider carefully whether to share data with researchers or not. Each application to data sharing request undergo strict assessment of staff in charge of research from the company behind the website, who usually have research background. As a result, access to data of chat-based services is limited to most researchers.\\
On the other hand, forums or discussion boards allow user to post content publicly, inviting other to comment the post with relevant advices and support. Since the posts are publicized on forums, any registered, subscribed users to that forums or even regular visitors can read the posts. While users of chat-based services feel more comfortable to share their stories along with many details, forums' users reach much wider variety of readers despite briefer version of the story. With growing number of people seeking help on online forums, researchers are able to collect data more easily and less worry about privacy issue than data from synchronous intervention.
\subsection*{Impact of computer science on suicide research}

chat -> private, user need confidentiality -> ethic -> not easy to get data
forum -> public, user seek help publicly from many other user on Internet -> focus on forum
suicide is big concern......




