%%
%% VERSION HISTORY
%%    22 May 2006 - John Papandriopoulos - Original version
%%    12 Jul 2007 - John Papandriopoulos - Converted into template
%%

\begin{abstract}
Over the past decade, the rise of social networking sites such as Twitter, Facebook, Reddit has changed the way people express their opinons. At the same time, the advancement of machine learning and natural language processing provides powerful tools to analyze massive amount of data users generated over time. Those tools are extremely beneficial to researchers in mental health field. The analysis of users' public posts help mental health professionals to arrange appropriate support for users with mental health issues, especially users who are vulnerable to suicidal thoughts.\\
Efforts have been made to utilize machine learning to predict suicidal individuals on social media platforms. However, classifiers built in those studies usually performed on a small randomly sampled subset of a big dataset obtained. Thus, the applicability of said classifiers in practice remains unknown. In this study, we try to build classifiers that produce state-of-the-art performance using a test set with small sample size. Then, we expand the test set to see whether the classifiers can maintain the performance on a more realistic scenario or not. We also report observations on language differences between labelled at-risk individuals and 'control' users. 
 

\end{abstract}

\clearpage
